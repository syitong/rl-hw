\begin{enumerate}
  \item This question will be checked 
  \item The value function returned by the \texttt{policy\_iter} function is
  \begin{minted}{python}
[[1. 1. 1. 1. 1.]
 [1. 1. 1. 1. 1.]
 [1. 1. 1. 1. 0.]
 [1. 1. 0. 1. 1.]]
  \end{minted}
  The corresponding policy is
  \begin{minted}{python}
[[0 0 0 0 0]
 [0 0 0 0 0]
 [0 0 0 0 0]
 [0 0 0 0 0]]
  \end{minted}
  So the robot at location 0 will keep trying going upward and
  never reach the exits. Because the only rewards at exits do
  not tell the robot to find the exits as soon as possible, the
  value function obtained in this way only reflects the possibility
  of reaching exits of the position.
  \item The value function returned by the \texttt{value\_iter}
  function is
  \begin{minted}{python}
[[0.4096 0.512  0.64   0.64   0.8   ]
 [0.512  0.64   0.8    0.8    1.    ]
 [0.64   0.8    1.     1.     0.    ]
 [0.8    1.     0.     1.     1.    ]]
\end{minted}
  The corresponding policy is
  \begin{minted}{python}
[[1 1 2 1 2]
 [1 1 2 1 2]
 [1 1 2 1 0]
 [1 1 0 3 0]]
  \end{minted}
  This policy achieves the goal of finding exits with the fewest
  moves. By setting the discount factor less than 1, if the robot
  tries to maximize the total reward, it has to reduce the number
  of moves to avoid the reward at exits discounted to much. So
  the discount factor helps it to achieve the goal.
\end{enumerate}
