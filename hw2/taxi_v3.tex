%!TEX root = ./homework2.tex
\noindent{Question 5} [35 or 40? pts] (Variant of taxi-v2)
In this problem, we will use Taxi-v3 openai environment, which is a variant
of Taxi-v2 adapted for our purpose. It describes a scenario where the taxi,
colored by yellow, needs to find the passenger at one of the corners of
the 5x5 grid, colored by blue, and take the passenger to the destination,
colored by magenta.

To use Taxi-v3, first you need to install openai gym module. And then
download \texttt{mytaxi.zip} and unzip it under your working directory.
The following code will load an \texttt{Taxi-v3} instance for you.
\begin{minted}{python}
import gym
import mytaxi

env = gym.make('Taxi-v3').unwrapped # expose all parameters to user
\end{minted}
For detailed usage of openai gym, please read their online document.
For our purpose, \texttt{env.nS,env.nA,env.P} tell us number of states,
number of actions and the full transition and reward information. You will
see that \texttt{env.P} has the same structure with the one you write for
the MDP problem. Several useful methods you will need:
\begin{description}
  \item[\texttt{env.reset()}] Reset the environment to a random initial
  state.
  \item[\texttt{env.step(action)}] Take an action and receive the reward
  and new state.
  \item[\texttt{env.render()}] Print out the illustration of current state.
\end{description}
Using this environment, you will be asked to explore several policy prediction
and optimal control algorithms.

\begin{enumerate}[(a)]
\item The file \texttt{policy} gives you a policy stored as a dictionary
where keys are states and values are corresponding probability distribution
over actions. Use policy evaluation algorithm you wrote for the MDP problem in
Homework 1 to evaluate the value function of the give policy.
Set the stopping criteria \texttt{theta=0.01} and the discount factor
\texttt{gamma=1}. This result will be used as the \emph{true value function} for
the given policy.
\item MC estimation of state values
\item on-policy first visit MC control for $\epsilon$-soft policies
\item Implement the TD(0) algorithm on Page 120 of the textbook. Use it to
evaluate the value function of the given policy.
\begin{minted}{python}
def td0(env,policy=give_policy,baseline=true_value_fn,
    gamma=1,alpha=0.1,episodes=1000):
    np.random.seed(3)
    env.seed(5)
    ...
    return rms, V
    '''
    env: The Taxi-v3 environment.
    baseline: Default to the true value function you obtain from Part (a).
    rms: Root mean square error w.r.t. baseline at the end of each episode.
    So it is an array of length equal to number of episodes.
    root mean square between two vectors a,b is defined by
    sqrt(sum((a-b)**2)/len(a)).
    V: The final value function evaluated by the algorithm. Initialize it
    to all 0 zeros.
    '''
\end{minted}
Plot the two figures in your homework. One is the plot of rms vs episodes
and the other one is the scatter plot of true value functions and the one
obtained from TD(0).
\item Implement the Q-learning algorithm on Page 131 of the textbook. Use it
to find the optimal policy for the Taxi-v3.
\begin{minted}{python}
def qlearn(env,gamma=1,alpha=0.9,epsilon=0.1,runs=20,episodes=500):
    np.random.seed(3)
    env.seed(5)
    ...
    return avgrew
    '''
    runs: Number of repetitions of the learning process.
    avgrew: Sum of rewards during one episode, averaged over 20 runs.
    Initialize the Q table to all 0 zeros for each run.
    '''
\end{minted}
Plot the avgrew vs episodes in your homework.
\end{enumerate}

\textcolor{blue}{Will include details on Sunday}
